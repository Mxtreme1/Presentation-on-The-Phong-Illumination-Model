\documentclass[a4paper]{scrartcl}%{{{
\usepackage{float}
\usepackage{tikz}
\usetikzlibrary{arrows,automata}
\usepackage{pgf}
\usepackage[utf8x]{inputenc} % this is needed for umlaut}s
\usepackage[ngerman]{babel} % this is needed for umlauts
\usepackage[T1,T5]{fontenc}    % this is needed for correct output of umlauts in pd
\usepackage{amssymb}
\usepackage{amsmath}
\usepackage{mathabx}
\usepackage{mathrsfs}
\usepackage{dsfont}
\usepackage{wasysym}
\usepackage{graphicx}
\usepackage{fancyhdr}
\usepackage{lastpage}
\usepackage{imakeidx}
\setlength{\parskip}{\medskipamount}
\setlength{\parindent}{0pt}
\usepackage{enumitem}
\usepackage{hyperref}
\usepackage{verbatim}

%%%%%%%%%%%%%%%%%%%%%%%%
% Kopf- und Fusszeilen %
%%%%%%%%%%%%%%%%%%%%%%%%
\pagestyle{fancy}
\lhead{
        Maximilian Roth:
}
\chead{Phong Illumination Model}
\rhead{
    \begin{tabular}{rr}
        \today{} \\
        Seite \thepage{} von \pageref{LastPage}
    \end{tabular}
}
\lfoot{}
\cfoot{}
\rfoot{} 

\title{Phong Illumination Model}
\author{Maximilian Christian Roth}


%%%%%%%%%%%%%%%%%%%%%%%%
% Anfang des Dokuments %
%%%%%%%%%%%%%%%%%%%%%%%%%
\begin{document}
%}}}

\maketitle
\tableofcontents

\newpage

\section{Einführung}%
\label{sec:einfuhrung}

    Das Phong-Illumination-Model oder Phong Beleuchtungsmodell ist ein Verfahren zur Berechnung der Intensität von Reflexionen auf der Oberfläche von Objekten,\\
    welches im Jahr 1975 von Bùi Tường Phong \footnotemark[1] vorgestellt wurde.\\
    \footnotetext[1]{Bùi Tường Phong: \url{https://en.wikipedia.org/wiki/Bui_Tuong_Phong}} 

\section{RGB-Vektoren}%
\label{sec:rgb_vektoren}
    In der Computergraphik werden verschiedene Werte, wie zum Beispiel - bei üblichen Implementationen des Phong-Illumination-Models -
    die Intensität von Licht und Materialkonstanten, die etwas über das Reflexionsverhalten eines Objekts aussagen, als RGB-Vektoren dargestellt.\\
    Diese  

\section{Phongs Reflexionstypen}%
\label{sec:phongs_reflexionstypen}

    \subsection{Für die Berechnung benötigte Werte}%
    \label{sub:fur_die_berechnung_benotigte_werte}
        Um die Reflexion für einen Punkt zu berechnen benötigen wir einige Hintergrundinformationen:\\
        \begin{table}[H]
        \caption{Wichtige Notationen}
        \label{table:vars}
        \begin{tabular}{l|l|l}
            \hline
            \multicolumn{1}{|c|}{Variable} & \multicolumn{1}{c|}{Bedeutung}                         & \multicolumn{1}{c|}{Typ} \\ \hline
            $\vec{P}$                              & Der betrachtete Punkt auf der Oberfläche eines Objekts & Vektor                   \\ \hline
            $\vec{N}$                              & Die Oberflächennormale des Punktes                     & Vektor                   \\ \hline
            $\vec{L}$                              & Der Vektor von P zur Lichtquelle                       & Vektor                   \\ \hline
            $\vec{R}$                              & Der reflektierte Lichtvektor am Punkt P                & Vektor                   \\ \hline
            $\vec{V}$                              & Der Vektor von P zum Viewpoint                         & Vektor                   \\ \hline
            $[I_{a}]$                    & Die Intensität des Umgebungslicht                      & RGB-Vektor               \\ \hline
            $[I_{in}]$                         & Die Intensität einer Lichtquelle                       & RGB-Vektor               \\ \hline
            $[I_{ambient}]$                         & Die Intensität der ambienten Lichtreflexion                       & RGB-Vektor               \\ \hline
            $[I_{diffus}]$                         & Die Intensität der diffusen Lichtreflexion                       & RGB-Vektor               \\ \hline
            $[I_{spekular}]$                         & Die Intensität der spekularen Lichtreflexion                       & RGB-Vektor               \\ \hline
            $[k_{ambient}]$                    & Die ambiente Materialkonstante                         & RGB-Vektor               \\ \hline
            $[k_{diffus}]$                     & Die diffuse Materialkonstante                          & RGB-Vektor               \\ \hline
            $[k_{spekular}]$                   & Die spekulare Materialkonstante                        & RGB-Vektor              \\ \hline
            n                   & Die Rauigkeit des Materials                        & Konstante              \\ \hline
            m                   & 'Shininess' im Blinn-Phong-Model                        & Konstante
        \end{tabular}
        \end{table}
    
    

    Im Phong-Illumination-Model \footnotemark[2] unterscheidet Bùi Tường Phong bei der Berechnung die Reflexion von Licht in folgende drei Subtypen:\\
    \footnotetext[2]{Paper: \url{http://www.cs.northwestern.edu/~ago820/cs395/Papers/Phong_1975.pdf}}

    \subsection{Die ambiente Reflexion}%
    \label{sub:die_ambiente_reflexion}
        Das Umgebungslicht, das von anderen Objekten im Raum and den betrachteten Punkt P (Siehe Tabelle ~\ref{table:vars}) reflektiert wird.\\
        Physikalisch gesehen müsste hier für jeden Punkt der Weg jedes Photons berechnet werden, um eine realistische Reflexion zu erreichen.\\
        Da dies jedoch sehr auswändig wäre wird einfach eine Umgebungslichtintensität für den Punkt gegeben mit dem dann die Reflexion berechnet wird.\\
        Die meisten Implementationen des Phong-Illumination-Models machen es sich jedoch noch einfacher und nehmen global die selbe Intensität an.\\
        \\Berechnet wird die Intensität der Reflexion des ambienten Lichts in der Regel wie folgt:\\
        \begin{equation}
            \label{eq:amb}
                [I_{ambient}] = [I_a] \cdot [k_{ambient}]\\
        \end{equation}
        Für die Variablen siehe Tabelle ~\ref{table:vars}\\

    \subsection{Die diffuse Reflexion}%
    \label{sub:die_diffuse_reflexion}
        Der zweite Typ, in den Phong Reflexion unterteilt ist die diffuse Reflexion.\\
        Sie beschreibt, wie Licht, das direkt von einer Lichtquelle auf das Objekt trifft absorbiert, und damit auch wie es reflektiert wird.\\
        Hier wird das von anderen Objekten auf den betrachteten Punkt reflektierte Licht nicht betrachtet.\\
        \\Die Berechnungsvorschrift lautet:\\
        \begin{equation}
            \label{eq:diff}
                [I_{diffus}] = [I_{in}] \cdot 
        \end{equation}
    
    




\end{document}
